\section{Uruchamianie}

Przed rozpoczęciem prac z~aplikacją należy pobrać niezbędne zależności. Na komputerze powinien zostać zainstalowany interpreter języka \textit{Python} oraz moduł \textit{pip}. Dostarczane przez producenta ESP8266 narzędzia wykorzystują Pythona w~wersji trzeciej. W~przypadku korzystania z~systemu operacyjnego Ubuntu konieczne może okazać się użycie aplikacji \verb|update-alternatives|, która podmieni standardowe polecenie \verb|python| (odnoszące się domyślnie do binariów wersji drugiej) na wymagany wariant \footnote{Przykład użycia: sudo update-alternatives --install /usr/bin/python python /usr/bin/python3.8 3}. Dodatkowo konieczne będzie zainstalowanie oprogramowania \verb|CMake|. Sam pakiet narzedzi dla ESP można pobrać z~repozytorium pod adresem https://github.com/espressif/ESP8266\_RTOS\_SDK. Przy pobieraniu warto użyć opcji \verb|--recusrive|. Repozytorium nie zawiera potrzebnego dla mikrokontrolera kompilatora. Aby go pozyskać należy wejść do pobranego katalogu i~wywołać polecenie \verb|./install.sh|. Wcześniejsze ustawienie zmiennej \verb|IDF_TOOLS_PATH| pozwala zdeterminować ścieżkę instalacji kompilatora. Tak przygotowane środowisko zawiera wszystkie elementy niezbędne do zbudowania aplikacji i~załadowania jej do pamięci węzła.

W~procesie budowania potrzebny jest dostęp do pewnych zmiennych środowiska. Ustawienie ich odbywa się poprzez wywołanie komendy \verb|source| na dwóch skryptach - \verb|export.sh| i~\verb|add_path.sh| - znajdujących się w~głównym folderze SDK. Ponadto należy ustawić zmienną \verb|PROJECT_HOME| na bezwzględną ścieżkę do folderu, w~którym znajdują się pliki projektu (np. \verb|/home/obir_repo/project|).

Po przygotowaniu środowiska aplikacja jest gotowa do zbudowania. Z~poziomu katalogu \verb|project/coap_server| należy wywołać polecenie \verb|idf.py build| \footnote{W~niektórych sytuacjach konieczne może być usunięcie plików tymczasowych z~poprzedniego budowania za pomoca polecenia idf.py fullclean}. Przesłanie obrazu binarnego do urządzenia odbywa się poprzez komendę \verb|idf.py flash|. \verb|idf.py monitor| uruchamia monitor portu szeregowego, na którym węzeł wypisuje informacje diagnostyczne. Opcjonalnie możliwe jest wywołanie \verb|idf.py flash monitor|, które wykona obie akcje sekwencyjnie.
