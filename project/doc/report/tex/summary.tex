\section{Podsumowanie}

Dzięki temu projektowi nauczyliśmy się nie tylko wiele na temat protokołu CoAP, ale też o strukturze typowego serwera w Internecie Rzeczy. Nasz pomysł z realizacją tego projektu nie na symulatorze Arduino, lecz na fizycznym module okazał się strzałem w dziesiątkę. Rozbudowane wymagania dotyczące całokształtu sprawiały, że projekt ten był dosć ambitny, lecz jak widać wykonalny. Dzięki temu mogliśmy też utrwalić nasze umiejętności obsługi ESP8266. Zdobyta wiedza i doświadczenie na pewno pomoga nam w dalszym rozwijaniu się w dziedzinie Internetu Rzeczy.